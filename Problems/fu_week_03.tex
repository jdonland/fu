\documentclass{article}
\usepackage[utf8]{inputenc}
\usepackage{fu}

\title{\textit{Forever Undecided} Reading Group \\ Week 3}
\date{}
\author{}
\begin{document}

\maketitle

\section*{Classical Propositional Logic}

\subsection*{Formulas, Subformulas, and Valuations}

\begin{definition}
    A \textit{formula} (of classical propositional logic, or $\CPL$) is an \textit{atom} (denoted by $p, q, r, s$, etc.), a formula preceded by $\neg$; or two formulas separated by $\wedge$, $\vee$, $\to$, or $\iff$ and enclosed in parentheses. Nothing else is a formula. 
\end{definition}

\begin{definition}    
    The symbols $\neg$, $\wedge$, $\vee$, $\to$, and $\iff$ are \textit{connectives}. They are called \textit{negation}, \textit{conjunction}, \textit{disjunction}, \textit{material implication}, and \textit{material equivalence}, respectively. A formula containing no connectives is \textit{atomic}.
\end{definition}

\begin{notation}
    If the outermost symbols of a formula are parentheses, we omit them.
\end{notation}

\begin{definition}
    If a formula $A$ is atomic, then the only \textit{subformula} of $A$ is $A$ itself. The subformulas of $\neg A$ are $\neg A$ together with all the subformulas of $A$. The subformulas of $A \wedge B$, $A \vee B$, etc., are the formula itself together with all the subformulas of $A$ and $B$. The atomic subformulas of a formula are its \textit{vocabulary}.
\end{definition}

\begin{prob}{3.1}
List all of the subformulas of each formula:
    \begin{multicols}{2}
    \begin{enumerate}[a)]
        \item $p$
        \item $p \wedge \neg q$
        \item $(r \vee s) \wedge t$
        \item $(q \wedge (\neg r \to s)) \iff (s \to (q \vee t))$
        \item
        \item
    \end{enumerate}
    \end{multicols}
\end{prob}

\begin{definition}
    A \textit{valuation} is a function from formulas to the \textit{truth values} $\{ \true, \false \}$ such that for any formulas $A$ and $B$:
    \begin{align*}
        v(\neg A) & = \begin{dcases} \false \textrm{ if } v(A) = \true, \\ \true \textrm{ otherwise} \end{dcases} \\
        v(A \wedge B) & = \begin{dcases} \false \textrm{ if } v(A) = \false \textrm{ or } v(B) = \false \textrm{ (or both)}, \\ \true \textrm{ otherwise} \end{dcases} \\
        v(A \vee B) & = \begin{dcases} \false \textrm{ if } v(A) = \false \textrm{ and } v(B) = \false, \\ \true \textrm{ otherwise} \end{dcases} \\
        v(A \to B) & = \begin{dcases} \false \textrm{ if } v(A) = \true \textrm{ and } v(B) = \false, \\ \true \textrm{ otherwise}  \end{dcases} \\
        v(A \iff B) & = \begin{dcases} \false \textrm{ if } v(A) \neq v(B), \\ \true \textrm{ otherwise} \end{dcases}
    \end{align*}
\end{definition}

\begin{proposition}
If $v$ and $w$ are valuations and $v(p) = w(p)$ for each atom $p$ in the vocabulary of $A$, then $v(A) = w(A)$.
\end{proposition}

\section*{Tautologies, Contradictions, and Contingencies}

\begin{definition}
    A formula $A$ is a \textit{tautology} if $v(A) = \true$ for all valuations $v$, a \textit{contradiction} if $v(A) = \false$ for all valuations $v$, and a \textit{contingency} otherwise.
\end{definition}

\begin{proposition}
A formula $A$ is a tautology (contradiction) if and only if $\neg A$ is a contradiction (tautology). $A$ is a contingency if and only if $\neg A$ is a contingency.
\end{proposition}

\section*{Truth Tables}

It can be determined whether a formula is a tautology, a contradiction, or a contingency by exhaustively testing every combination of $\true$ and $\false$ for all of the atoms in its vocabulary. It helps to do this by breaking the formula into its subformulas and keeping track of the results in a \textit{truth table}.

For example, the following truth table demonstrates that $((p \to q) \to p) \to p$ is a tautology:

\begin{center}
\begin{tabular}{c|c||c|c|c}
    $p$ & $q$ & $p \to q$ & $(p \to q) \to p$ & $((p \to q) \to p) \to p$ \\
    \hline
    $\true$  & $\true$  & $\true$  & $\true$  & $\true$ \\
    $\true$  & $\false$ & $\false$ & $\true$  & $\true$ \\
    $\false$ & $\true$  & $\true$  & $\false$ & $\true$ \\
    $\false$ & $\false$ & $\true$  & $\false$ & $\true$ \\
\end{tabular}
\end{center}

Moreover, a truth table shows which combinations render a contingency true and which render it false. For example, this truth table shows that the contingency $\neg r \to (t \vee s)$ is \textbf{true} for all valuations on which at least one of $r$, $t$, or $s$ is \textbf{true}:

\begin{center}
\begin{tabular}{c|c|c||c|c|c}
    $r$ & $t$ & $s$ & $\neg r$ & $t \vee s$ & $\neg r \to (t \vee s)$ \\
    \hline
    $\true$  & $\true$  & $\true$  & $\false$ & $\true$  & $\true$  \\
    $\true$  & $\true$  & $\false$ & $\false$ & $\true$  & $\true$  \\
    $\true$  & $\false$ & $\true$  & $\false$ & $\true$  & $\true$  \\
    $\true$  & $\false$ & $\false$ & $\false$ & $\false$ & $\true$  \\
    $\false$ & $\true$  & $\true$  & $\true$  & $\true$  & $\true$  \\
    $\false$ & $\true$  & $\false$ & $\true$  & $\true$  & $\true$  \\
    $\false$ & $\false$ & $\true$  & $\true$  & $\true$  & $\true$  \\
    $\false$ & $\false$ & $\false$ & $\true$  & $\false$ & $\false$ \\
\end{tabular}
\end{center}

\begin{prob}{3.2}
    Classify each of the following formulas as a tautology, contradiction, or contingency:
    \begin{multicols}{2}
    \begin{enumerate}[a)]
    \item $(p \iff r) \wedge (\neg q \to \neg r)$
    \item $p \vee \neg p$
    \item $p \wedge \neg p$
    \item $((p \to q) \to p) \to p$
    \item $(p \iff (\neg p \vee \neg q)) \to p$
    \item $q$
    \item $(r \to s) \vee (s \to r)$
    \item $(p \to q) \wedge (p \wedge \neg q)$
    \end{enumerate}
    \end{multicols}
\end{prob}

\end{document}