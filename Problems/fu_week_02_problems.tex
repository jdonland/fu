\documentclass{article}
\usepackage[utf8]{inputenc}
\usepackage{fu}

\title{\emph{Forever Undecided} Reading Group \\ Week 2}
\date{}
\author{}
\begin{document}

\maketitle

\begin{definition}
    A \emph{formula} of classical propositional logic (hereafter $\CPL$) is an \emph{atom} (denoted by $p, q, r, s$, etc.), a formula preceded by $\neg$; or two formulas separated by $\wedge$, $\vee$, $\to$, or $\iff$ and enclosed in parentheses. Nothing else is a formula. The set of all such formulas is denoted $\mathsf{Form}_\CPL$.
\end{definition}

\begin{notation}
    If the outermost symbols of a formula are parentheses, we omit them.
\end{notation}

\begin{definition}
    A \emph{valuation} is a function $v:\mathsf{Form}_\CPL \to \{ \true, \false \}$ such that for any formulas $A$ and $B$:
    \begin{align*}
        v(\neg A) & = \begin{dcases} \false \textrm{ if } v(A) = \true, \\ \true \textrm{ otherwise} \end{dcases} \\
        v(A \wedge B) & = \begin{dcases} \false \textrm{ if } v(A) = \false \textrm{ or } v(B) = \false \textrm{ (or both)}, \\ \true \textrm{ otherwise} \end{dcases} \\
        v(A \vee B) & = \begin{dcases} \false \textrm{ if } v(A) = \false \textrm{ and } v(B) = \false, \\ \true \textrm{ otherwise} \end{dcases} \\
        v(A \to B) & = \begin{dcases} \false \textrm{ if } v(A) = \true \textrm{ and } v(B) = \false, \\ \true \textrm{ otherwise}  \end{dcases} \\
        v(A \iff B) & = \begin{dcases} \false \textrm{ if } v(A) \neq v(B), \\ \true \textrm{ otherwise} \end{dcases}
    \end{align*}
\end{definition}

\begin{proposition}
A valuation is uniquely determined by its action on atoms.
\end{proposition}

\begin{definition}
    A formula $A$ is a \emph{tautology} of $\CPL$ if $v(A) = \true$ for all valuations $v$, a \emph{contradiction} if $v(A) = \false$ for all valuations $v$, and a \emph{contingency} otherwise.
\end{definition}

\begin{proposition}
A formula $A$ is a tautology (contradiction) if and only if $\neg A$ is a contradiction (tautology). $A$ is a contingency if and only if $\neg A$ is a contingency.
\end{proposition}

\begin{proposition}
To demonstrate that a formula $A$ is not a tautology (contradiction) it suffices to identify any valuation $v$ in which $v(A) = \false$ ($\true$). To demonstrate that $A$ is a contingency, it suffices to identify two valuations $v$ and $w$ such that $v(A) \neq w(A)$.
\end{proposition}

\begin{prob}{2.1}
    Determine whether each of the following formulas is a tautology, a contradiction, or a contingency.
    \begin{multicols}{2}
    \begin{enumerate}[a)]
    \item $(p \iff r) \wedge (\neg q \to \neg r)$
    \item $p \vee \neg p$
    \item $p \wedge \neg p$
    \item $((p \to q) \to p) \to p$
    \item $(p \iff (\neg p \vee \neg q)) \to p$
    \item $(p \to q) \wedge (p \wedge \neg q)$
    \end{enumerate}
    \end{multicols}
\end{prob}

\begin{definition}
    Let $\Gamma = \{A_1, A_2, \dots A_n \}$ and $\Delta = \{B_1, B_2, \dots B_m \}$ be sets of formulas of $\CPL$. If there is no valuation $v$ such that $v(A_i) = \true$ for all $A_i \in \Gamma$, we write $\Gamma \models_\CPL$. If for all valuations $v$, there exists some $B_j \in \Delta$ such that $v(B_j) = \true$, we write $\models_\CPL \Delta$. Finally, if, for all valuations $v$ such that $v(A_i) = \true$ for all $A_i \in \Gamma$, there exists some $B_j \in \Delta$ such that $v(B_j) = \true$, we write $\Gamma \models_\CPL \Delta$. 
\end{definition}

\begin{notation} 
    Let $\Gamma$ and $\Delta$ be sets of formulas and let $A$ and $B$ be formulas. We abbreviate $\Gamma \cup \Delta$ as $\Gamma, \Delta$; $\Gamma \cup \{ A \}$ as $\Gamma, A$; and $\{ A \} \cup \{ B \}$ as $A, B$.
\end{notation}

\begin{prob}{2.2}
    Show that:
    \begin{multicols}{2}
    \begin{enumerate}[a)]
    \item $\neg p \models_\CPL (q \to p) \to \neg q$
    \item ${r \to p}, {q \to \neg p}, q \models_\CPL \neg r$
    \item $(p \vee q) \vee r, \neg p, \neg q \models_\CPL r, s, t$
    \item ${p \iff r}, {\neg q \iff \neg r} \models_\CPL {p \iff q}$
    \item ${r \vee \neg s}, s, \neg t, {p \iff t} \models_\CPL {r \wedge \neg p}$
    \item $p, q, r, s \models_\CPL {r \to s}, {p \wedge \neg p}$
    \item ${\neg (p \wedge r)}, {s \to r}, {\neg p \to \neg s} \models_\CPL t, {\neg s}$
    \item ${\neg (p \iff q)}, {\neg t \to p}, {r \vee q} \models_\CPL {\neg r \to (t \vee s)}$
    \item ${\neg (p \wedge q)}, {\neg p \to r}, {\neg r \to q} \models_\CPL r$
    \end{enumerate}
    \end{multicols}
\end{prob}

\begin{definition}
    Let $A$ and $B$ be formulas of $\CPL$. If $A \models_\CPL B$ and $B \models_\CPL A$, we say $A$ and $B$ are \emph{logically equivalent}.
\end{definition}

\begin{prob}{2.3}
    Show that each pair of formulas is logically equivalent:
    \begin{multicols}{2}
    \begin{enumerate}[a)]
    \item $p$ and $\neg \neg p$
    \item $\neg (p \wedge q)$ and $\neg p \vee \neg q$
    \item $\neg (p \vee q)$ and $\neg p \wedge \neg q$
    \item $(p \to q) \wedge (q \to p)$ and $p \iff q$
    \item $p \to q$ and $\neg p \vee q$
    \item $p \to q$ and $\neg q \to \neg p$
    \end{enumerate}
    \end{multicols}
\end{prob}

\begin{proposition}
    Let $A$, $B$, and $C$ be formulas of $\CPL$. Then $(A \wedge B) \wedge C$ and $A \wedge (B \wedge C)$ are logically equivalent, and so are $(A \vee B) \vee C$ and $A \vee (B \vee C)$.
\end{proposition}

\begin{notation}
    In light of the previous proposition, we write, e.g., $A_1 \wedge A_2 \wedge \cdots \wedge A_n$ to indicate $A_1, A_2, \dots A_n$ connected by $\wedge$ in any order, and likewise with $\vee$. 
\end{notation}

\begin{prob}{2.4}
    Show that:
    \begin{enumerate}[a)]
    \item If $A$ and $B$ are logically equivalent, then $\models_\CPL A \iff B$.
    \item If $\Gamma \models_\CPL \Delta$ then $\Gamma, A \models_\CPL \Delta$ and $\Gamma \models_\CPL \Delta, A$.
    \item $A, \neg A \models_\CPL$ and $\models_\CPL A, \neg A$.
    \item $A_1, A_2, \dots A_n \models_\CPL B_1, B_2, \dots B_m$ if and only if $A_1 \wedge A_2 \wedge \cdots \wedge A_n \models_\CPL B_1 \vee B_2 \vee \cdots \vee B_m$.
    \item $\Gamma \models_\CPL \Delta, A \to B$ if and only if $\Gamma, A \models_\CPL \Delta, B$.
    \end{enumerate}
\end{prob}

\begin{prob}{2.5 (Bonus)}
    Prove that:
    \begin{enumerate}[a)]
    \item If $A \models_\CPL B$, and $A$ and $B$ share no atoms in common, then $A$ is a contradiction or $B$ is a tautology. (Or both.)
    \item For all $k \geq 1$, logical equivalence partitions the subset of $\mathsf{Form_{CPL}}$ in which all and only the atoms $p_1, p_2, \dots, p_k$ appear into $2^k$ classes.
    \item Every formula of $\CPL$ is logically equivalent to a formula of the form $A_1 \wedge A_2 \wedge \cdots \wedge A_n$ where each $A_i$ is a formula of the form $B_1 \vee B_2 \vee \cdots \vee B_m$ and each $B_j$ is an atom or an atom preceded by $\neg$ (\emph{conjunctive normal form}), and likewise with $\wedge$ and $\vee$ swapped (\emph{disjunctive normal form}).
    \end{enumerate}
\end{prob}

\end{document}
