\documentclass{article}
\usepackage[utf8]{inputenc}
\usepackage{fu}

% to do: correct rules for equivalence

\title{\emph{Forever Undecided} Reading Group \\ Week 3}
\date{}
\author{}
\begin{document}

\maketitle

It can be determined whether a formula of $\CPL$ is a tautology, a contradiction, or a contingency by exhaustively testing every combination of $\true$ and $\false$ for all of the atoms which appear in it. It helps to do this by breaking the formula into its subformulas and keeping track of the results in a table. These tables are known as \emph{truth tables}.

For example, the following truth table demonstrates that $((p \to q) \to p) \to p$ is a tautology:

\begin{center}
\begin{tabular}{c|c||c|c|c}
    $p$ & $q$ & $p \to q$ & $(p \to q) \to p$ & $((p \to q) \to p) \to p$ \\
    \hline
    $\true$  & $\true$  & $\true$  & $\true$  & $\true$ \\
    $\true$  & $\false$ & $\false$ & $\true$  & $\true$ \\
    $\false$ & $\true$  & $\true$  & $\false$ & $\true$ \\
    $\false$ & $\false$ & $\true$  & $\false$ & $\true$ \\
\end{tabular}
\end{center}

Moreover, a truth table shows which combinations render a contingency true and which render it false. For example, this truth table shows that the contingency $\neg r \to (t \vee s)$ is logically equivalent to $r \vee t \vee s$:

\begin{center}
\begin{tabular}{c|c|c||c|c|c}
    $r$ & $t$ & $s$ & $\neg r$ & $t \vee s$ & $\neg r \to (t \vee s)$ \\
    \hline
    $\true$  & $\true$  & $\true$  & $\false$ & $\true$  & $\true$  \\
    $\true$  & $\true$  & $\false$ & $\false$ & $\true$  & $\true$  \\
    $\true$  & $\false$ & $\true$  & $\false$ & $\true$  & $\true$  \\
    $\true$  & $\false$ & $\false$ & $\false$ & $\false$ & $\true$  \\
    $\false$ & $\true$  & $\true$  & $\true$  & $\true$  & $\true$  \\
    $\false$ & $\true$  & $\false$ & $\true$  & $\true$  & $\true$  \\
    $\false$ & $\false$ & $\true$  & $\true$  & $\true$  & $\true$  \\
    $\false$ & $\false$ & $\false$ & $\true$  & $\false$ & $\false$ \\
\end{tabular}
\end{center}

\begin{prob}{3.1}
    Consider the formula $(p \wedge q \to \neg r) \wedge (q \to s)$.
    \begin{enumerate}[a)]
        \item Write its truth table.
        \item Find a formula in conjunctive normal form and another in disjunctive normal form, each of which are logically equivalent to it.
    \end{enumerate}
\end{prob}

Unfortunately, if a formula contains $k$ distinct atoms then there are $2^k$ possibilities to check. When $k$ is large, this may be impractical. For logics other than $\CPL$, e.g., those with infinitely many truth values rather than just two, this process may not be possible at all.

\begin{definition}
    A \emph{sequent} (for our purposes) is an expression of the form $\Gamma \triangleright \Delta$ where $\Gamma$ and $\Delta$ are (possibly empty) sets of formulas.
\end{definition}

\begin{notation} 
    As before, we abbreviate $\Gamma \cup \Delta$ as $\Gamma, \Delta$; $\Gamma \cup \{ A \}$ as $\Gamma, A$; and $\{ A \} \cup \{ B \}$ as $A, B$.
\end{notation}

\begin{definition}
    A sequent is \emph{derivable} if it is of the form \sequent{A}{A} or if it can obtained from derivable sequents according to the following \emph{rules}, where the horizontal bar indicates that the sequent below may be obtained from the sequent(s) above.
\end{definition}

\[ \arraycolsep = 2em \def\arraystretch{3}
\begin{array}{c c}
% weakening
\sequentrule{\Gamma}{\Delta}{\Gamma, A}{\Delta} &
\sequentrule{\Gamma}{\Delta}{\Gamma}{\Delta, A} \\
% negation
\sequentrule{\Gamma, A}{\Delta}{\Gamma}{\Delta, \neg A} &
\sequentrule{\Gamma}{\Delta, A}{\Gamma, \neg A}{\Delta} \\
% conjunction
\sequentrule{\Gamma, A, B}{\Delta}{\Gamma, A \wedge B}{\Delta} &
\doublesequentrule{\Gamma}{\Delta, A}{\Gamma}{\Pi, B}{\Gamma}{\Delta, \Pi, A \wedge B} \\
% disjunction
\doublesequentrule{\Gamma, A}{\Delta}{\Gamma, B}{\Pi}{\Gamma, A \vee B}{\Delta, \Pi} & 
\sequentrule{\Gamma}{\Delta, A, B}{\Gamma}{\Delta, A \vee B} \\
% implication
\doublesequentrule{\Gamma}{\Delta, A}{\Gamma, B}{\Pi}{\Gamma, A \to B}{\Delta, \Pi} &
\sequentrule{\Gamma, A}{\Delta, B}{\Gamma}{\Delta, A \to B}
\end{array}
\]
\vspace{-1em}
\[ \arraycolsep = 2em \def\arraystretch{3}
\begin{array}{c c c}
\doublesequentrule{\Gamma, A}{\Delta, B}{\Pi, B}{\Theta, A}{\Gamma, \Pi}{\Delta, \Theta, A \iff B} &
\doublesequentrule{\Gamma}{\Delta, A}{\Pi, B}{\Theta}{\Gamma, \Pi, A \iff B}{\Delta, \Theta} &
\doublesequentrule{\Gamma}{\Delta, B}{\Pi, A}{\Theta}{\Gamma, \Pi, A \iff B}{\Delta, \Theta}
\end{array}
\]




\begin{proposition}
    Since the items on either side of a sequent are sets, we may interchange $\Gamma, A, A$ with $\Gamma, A$ and $\Gamma, A, B, \Delta$ with $\Gamma, B, A, \Delta$ whenever it is convenient.
\end{proposition}

Here is a derivation of \sequent{\neg p}{(q \to p) \to \neg q}:

\begin{prooftree}
    \AxiomC{\sequent{q}{q}}
    \UnaryInfC{\sequent{q}{p, q}}
    \UnaryInfC{\sequent{}{\neg q, p, q}}
    \AxiomC{\sequent{p}{p}}
    \UnaryInfC{\sequent{p}{\neg q, p}}
    \BinaryInfC{\sequent{q \to p}{\neg q, p}}
    \UnaryInfC{\sequent{\neg p, q \to p}{\neg q}}
    \UnaryInfC{\sequent{\neg p}{(q \to p) \to \neg q}}
\end{prooftree}

\begin{notation}
    If $\Gamma \triangleright \Delta$ is derivable, then we write $\Gamma \vdash \Delta$.
\end{notation}

\begin{proposition}
    $\Gamma \models \Delta$ if and only if $\Gamma \vdash \Delta$.
\end{proposition}

\end{document}