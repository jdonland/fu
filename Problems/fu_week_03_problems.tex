\documentclass{article}
\usepackage[utf8]{inputenc}
\usepackage{fu}

\title{\emph{Forever Undecided} Reading Group \\ Week 3}
\date{}
\author{}
\begin{document}

\maketitle

It can be determined whether a formula of $\CPL$ is a tautology, a contradiction, or a contingency by exhaustively testing every combination of $\true$ and $\false$ for all of the atoms which appear in it. It helps to do this by breaking the formula into its subformulas and keeping track of the results in a \emph{truth table}.

For example, the following truth table demonstrates that $((p \to q) \to p) \to p$ is a tautology:

\begin{center}
\begin{tabular}{c|c||c|c|c}
    $p$ & $q$ & $p \to q$ & $(p \to q) \to p$ & $((p \to q) \to p) \to p$ \\
    \hline
    $\true$  & $\true$  & $\true$  & $\true$  & $\true$ \\
    $\true$  & $\false$ & $\false$ & $\true$  & $\true$ \\
    $\false$ & $\true$  & $\true$  & $\false$ & $\true$ \\
    $\false$ & $\false$ & $\true$  & $\false$ & $\true$ \\
\end{tabular}
\end{center}

Moreover, a truth table shows which combinations render a contingency true and which render it false. For example, this truth table shows that the contingency $\neg r \to (t \vee s)$ is logically equivalent to $r \vee t \vee s$:

\begin{center}
\begin{tabular}{c|c|c||c|c|c}
    $r$ & $t$ & $s$ & $\neg r$ & $t \vee s$ & $\neg r \to (t \vee s)$ \\
    \hline
    $\true$  & $\true$  & $\true$  & $\false$ & $\true$  & $\true$  \\
    $\true$  & $\true$  & $\false$ & $\false$ & $\true$  & $\true$  \\
    $\true$  & $\false$ & $\true$  & $\false$ & $\true$  & $\true$  \\
    $\true$  & $\false$ & $\false$ & $\false$ & $\false$ & $\true$  \\
    $\false$ & $\true$  & $\true$  & $\true$  & $\true$  & $\true$  \\
    $\false$ & $\true$  & $\false$ & $\true$  & $\true$  & $\true$  \\
    $\false$ & $\false$ & $\true$  & $\true$  & $\true$  & $\true$  \\
    $\false$ & $\false$ & $\false$ & $\true$  & $\false$ & $\false$ \\
\end{tabular}
\end{center}

\begin{prob}{3.1}
    Consider the formula $(p \wedge q \to \neg r) \wedge (q \to s)$.
    \begin{enumerate}[a)]
        \item Write its truth table.
        \item Find a formula in conjunctive normal form and another in disjunctive normal form, each of which are logically equivalent to it.
    \end{enumerate}
\end{prob}

Unfortunately, if a formula contains $k$ distinct atoms then there are $2^k$ possibilities to check. When $k$ is large, this may be impractical. For logics other than $\CPL$, e.g., those with infinitely many truth values rather than just two, this process may not be possible at all.

\begin{definition}
    A \emph{proof} is 
\end{definition}

\begin{notation}
    If there is a proof with premisses $\Gamma$ and conclusion $B$ then we write $\Gamma \vdash B$.
\end{notation}

\begin{prob}{3.2}
For each of \textbf{2.2} a)--i), find a proof with all of the formulas on the left of the $\vDash$ as premisses and the formula on the right as its conclusion.
\end{prob}

\begin{proposition}
    $\Gamma \models B$ if and only if $\Gamma \vdash B$.
\end{proposition}

\end{document}